\section{实验总结}
这次课程设计的主要原理是利用音频的幅频特性关系在频域将噪音信号去除,主要考察的是信号幅频特性以及滤波器的相关应用。\\
\indent 本设计的去除蜂鸣效果很好,且考虑到低频信号仍包含重要的人声要素,不可使用滤波器直接去除的问题。\textbf{可以将绝大部分的蜂鸣噪音全部去除,并且极高质量地保存了人声语言音频}\textit{【蜂鸣噪声绝大多数被消除且人声语言贴近真实声音、无明显‘失真’】},可以清晰地听到男声语音为:\textbf{“这里是电子科技大学”}。不过在音频开头、结尾仍有一小部分杂音没有去除,这或许是可以进一步改进的地方、但个人判断认为此声音可能是录音开启与结束的操作音直接存在于原始音频中,故而难以被滤除。\\
\indent 同时,本设计的功能相对更加开放,提供了许多可改变参数的参数,故此设计可以面向其他类似的音频使用,具有极大的扩展性。