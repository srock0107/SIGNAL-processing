\section{实验目的}
\begin{table}[!ht]
    \centering
    \begin{tabular}{|l|}
    \hline
        电梯处于1楼,按KEY3,LED3亮,电梯处于上行状态,楼层显示1;运行5秒至2楼后LED3灭,2楼待机。 \\ \hline
        电梯处于2楼,按KEY2,LED2亮,电梯处于下行状态,楼层显示2;运行5秒至1楼后LED2灭,1楼待机。 \\ \hline
        电梯处于1楼,按KEY1,LED1亮,电梯处于上行状态,楼层显示1;电梯运行至2楼后LED1灭,2楼待机。 \\ \hline
        电梯处于2楼,按KEY0,LED0亮,电梯处于下行状态,楼层显示2;运行至1楼后LED0灭,1楼待机。 \\ \hline
        电梯处于1楼,按KEY0或KEY2,指示灯均不应该亮,电梯1楼待机。 \\ \hline
        电梯处于1楼,按KEY3,LED3亮,电梯处于上行状态时,立刻按 KEY2,LED2亮,电梯继续运行至2楼,LED3灭;然后电梯自动返回1楼待机,LED2灭。 \\ \hline
        电梯处于1楼;按KEY3,LED3亮,电梯处于上行状态时,立刻按 KEY0,LED0亮,电梯继续运行至2楼,LED3灭;然后电梯自动返回1楼待机,LED0灭。 \\ \hline
        电梯处于1楼,按KEY1或 KEY3,让电梯运行至2楼待机;再按KEY1/ KEY3,指示灯均不应该亮,电梯继续待机。 \\ \hline
        电梯处于2楼,按KEY2,LED2亮,电梯处于下行状态时,立刻按 KEY3,LED3亮,电梯继续运行至1楼,LED2灭;然后电梯自动返回2楼待机,LED3灭。 \\ \hline
        电梯处于2楼,按KEY2,LED2亮,电梯处于下行状态时,立刻按 KEY1,LED1亮,电梯继续运行至1楼,LED2灭;然后电梯自动返回2楼待机,LED1灭。 \\ \hline
        将复位开关SW11拨至下方,电梯运行至1楼(运行时间5S)。 \\ \hline
        将启动开关SW0拨至下方,按任意按键均无反应。 \\ \hline
    \end{tabular}
\end{table}