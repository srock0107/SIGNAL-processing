
\section*{代码附录}

\definecolor{vgreen}{RGB}{104,180,104}
\definecolor{vblue}{RGB}{49,49,255}
\definecolor{vorange}{RGB}{255,143,102}
\lstdefinestyle{verilog-style}
{
    language=Verilog,
    basicstyle=\ttfamily,
    keywordstyle=\color{vblue},
    identifierstyle=\color{black},
    commentstyle=\color{vgreen},
    numbers=left,
    frame=shadowbox,
    breaklines=true, 
    numberstyle={\tiny \color{black}},
    numbersep=10pt,
    tabsize=8
}
\begin{lstlisting}[
    style={verilog-style}]
//1:顶层模块===============================================
 module top_module(
  input clk,
  input start,    
  input reset, 
  input key0,key1,key2,key3,
  output row,
  output [3:0]led,
  output [5:0]dig,
  output [7:0]seg,
  output beep,
  output [4:0]led_b);
    assign row=1'b0;
    wire clk_1k;
    wire clk_1s;
    wire clk_20ms;

  fenpin u1(.clk(clk),
         .clk_1k(clk_1k),
         .clk1sout(clk_1s),
         .clk_20ms(clk_20ms));
         
         wire k0;
         wire k1;
         wire k2;
         wire k3;
  key_debounce u2(
         .clk(clk_20ms),
         .key_in(key0),
         .key_out(k0));
  key_debounce u3(
         .clk(clk_20ms),
         .key_in(key1),
         .key_out(k1));
  key_debounce u4(
         .clk(clk_20ms),
         .key_in(key2),
         .key_out(k2));
  key_debounce u5(
         .clk(clk_20ms),
         .key_in(key3),
         .key_out(k3));    
  wire [1:0]state;
  wire [1:0]floor;      
  operation u6(
         .clk(clk),
         .set(start),
         .reset(reset),
         .key0(k0),
         .key1(k1),
         .key2(k2),
         .key3(k3),
         .led(led),
         .state(state),
         .floor(floor)); 
  display u7(
         .floor(floor),
         .state(state),
         .clk(clk_1k),
         .seg(seg),
         .clk_1s(clk_1s),
         .led_b(led_b),
         .dig(dig)); 
  beep u8(
         .clk(clk),
         .state(state),
         .floor(floor),
         .beep(beep));                                          
endmodule

//2.分频器===============================================
module fenpin(
    input clk,clr,
    output reg clk_1k=0,  //1kHz信号
    output reg clk1sout=0,//1Hz信号
    output reg clk_20ms=0,//50Hz信号
    integer    clk1s_cnt=0
    );
 reg[24:0] cnt=0;
 always @ (posedge clk)//分频得到1kHz信号
    begin
      if (cnt==25000)
        begin
           clk_1k=~clk_1k; cnt=0;
         end
      else  cnt=cnt+1;
    end

reg [31:0] temp = 0;
always@(posedge clk)//分频得到50Hz信号
    begin
      if(temp == 999999)
      begin
            temp <= 0;clk_20ms <= ~clk_20ms;
      end
      else  temp <= temp+1;
      end
 
 always@(posedge clk)//分频得到1Hz信号
 begin    
    if(clr)    
    begin   
           clk1s_cnt<=0; clk1sout<=0;  
    end 
    else  if(clk1s_cnt==24999999) 
    	begin
     	   clk1s_cnt<=0;clk1sout<=~clk1sout;
     	end
    else   clk1s_cnt<=clk1s_cnt+1;
 end
endmodule

//3.按键消抖模块=====================================
module key_debounce(
    input key_in,
    input clk,
    output key_out);
    reg  btn0=0;
    reg  btn1=0;
    reg  btn2=0;

always@(posedge clk)
  begin
    btn0<=key_in;
    btn1<=btn0;
    btn2<=btn1;
  end
 
assign key_out=((btn0&btn1)&(~btn2) | (btn0&btn1&btn2) | ((~btn0)&btn1&btn2));
endmodule

//4.状态控制模块=======================================
module operation(
    input clk,
    input set,
    input reset,
    input key0,
    input key1,
    input key2,
    input key3,
    output reg [3:0]led=0,//指示灯
    output reg [1:0]state=0,//0待机状态;1上行;2下行
    output reg [1:0]floor=1  //1一楼;2二楼 
	);
    reg [27:0]cnt=0; 
    reg [27:0]clk_count=0;
    reg k0=0;
    reg k1=0;
    reg k2=0;
    reg k3=0; 
    reg [1:0] first=0; 
always@(posedge clk)
  begin
    if(!reset)//复位开关被拨回
       begin
           if(floor==2)//如果电梯在2楼,则运行五秒后回到一楼
             if(clk_count==249999999) //5s计时
             begin            
                clk_count=0;  
                floor=1;    //1楼                  
                state=0;    //待机状态       
                led=4'b0000;//指示灯不亮
              end
             else 
             begin 
                clk_count=clk_count+1;
                state=2;    //下行状态
                led=4'b0000;//指示灯不亮
             end
       end
   else if(set)//复位开关未被拨下,且启动start有效
   begin
     if(led)//当led非0,即电梯处于运行状态时
       begin 
         if(clk_count==249999999)//5s计时 
          begin            
            clk_count=0;                        
            if(floor==1)  
                begin floor=floor+1; 
                end//计时五秒后从一楼切换到二楼
            else        
                begin floor=floor-1; 
                end//计时五秒后从二楼切换到一楼       
            
            //key3,key2接连被按下,到二楼后保持led2亮起,切换到下行状态
            if((k3&k2)&&first==1) 
                begin state=2;k3=0;k2=0;led[3]=0;first=0;
                end

            //key3,key0接连被按下,到二楼后保持led0亮起,切换到下行状态
            else if(k3&k0) 
                begin state=2;k3=0;k0=0;led[3]=0;
                end

            //key2,key3接连被按下,到一楼后保持led3亮起,切换到上行状态
            else if((k2&k3)&&first==2) 
                begin state=1;k2=0;k3=0;led[2]=0;first=0;
                end

            //key2,key1接连被按下,到一楼后保持led1亮起,切换到上行状态
            else if(k2&k1) 
                begin state=1;k2=0;k1=0;led[2]=0;
                end

            //回到待机状态
            else  
                begin state=0;led=4'b0000;k0=0;k1=0;k2=0;k3=0;
                end 
          end 

         else  clk_count=clk_count+1;
    //在运行过程中保持指示灯正常亮起,并且在按键在电梯运行过程中被按下时进行记录
     if(key2|k2)
        begin k2=1;k0=0;led[2]=1;
        end
     if(key0|k0) 
        begin k0=1;k2=0;led[0]=1;
        end
     if(key3|k3) 
        begin k3=1;k1=0;led[3]=1;
        end
     if(key1|k1) 
        begin k1=1;k3=0;led[1]=1;
        end
    end
        
    else if((key0)&&floor==2)  //当前在二楼,一楼按key0
            begin
                led[0]=1; 
                state=2; //下行                         
            end
                                
    else if((key1)&&floor==1)  //当前在一楼,二楼按key1
            begin
                led[1]=1;
                state=1;//上行
            end
     
    else if((key2)&&floor==2)  //当前在二楼 电梯内按下 
            begin
                led[2]=1;
                state=2; //下行
                k2=1;
                first=2;//用于表明key2先于key3被按下
            end 
                                 
    else if((key3)&&floor==1)  //当前在一楼 电梯内按上  
            begin
                led[3]=1;
                state=1; //上行 
                k3=1;
                first=1;//用于表明key3先于key2被按下
            end
        end
    end
endmodule

//5.译码显示及流水指示灯模块==========================

module display(
    input [1:0]floor,//楼层
    input [1:0]state,//状态
    input clk,
    input clk_1s,
    output reg [4:0] led_b=0,
    output reg [7:0] seg,
    output reg [5:0] dig
);
reg num=0;
	always@(posedge clk)
	begin
		if (num==1) num=0;
		else num=num+1;
	end
   //位选
	always@(num)
	begin	
		case(num)
		0:dig=6'b111110;
		1:dig=6'b111101;
		default: dig=0;
		endcase
	end
	
always@(posedge clk_1s or posedge state or negedge state)
begin
	if(state==0)
	    led_b[4:0]=5'b00000;
    else if (state ==1)//电梯处于上行状态
        begin
            if (led_b[4:0]==5'b00000)
                led_b[4:0]=5'b10000;
        else
                led_b[4:0]=led_b[4:0]>>1; 
    //向右移一位。10000-01000-00100-00010-00001-00000
        end
    else if (state ==2)//电梯处于下行状态
        begin
            if (led_b[4:0]==5'b00000)
                led_b[4:0]=5'b00001;//到了00000,下一个是00001
            else
                led_b[4:0]=led_b[4:0]<<1; 
    //向左移一位。00001-00010-00100-01000-10000-00000
        end
    else led_b[4:0]=5'b00000;
	end
	
//选择器,确定显示数据
reg [3:0] disp_data;
always@(num)
begin	
	case(num)
	0:disp_data=floor;
	1:disp_data=state+3'b011;//3待机 4上行 5下行
	default: disp_data=0;
	endcase
end	
//显示译码器
always@(disp_data)
	begin
		case(disp_data)
		4'h1: seg=8'h06;//1楼
		4'h2: seg=8'h5b;//2楼
		4'h3: seg=8'h40;//待机
		4'h4: seg=8'h01;//上行
		4'h5: seg=8'h08;//下行
		default: seg=0;
		endcase
	end
endmodule

//6.蜂鸣==================================
module beep(
    input clk,
    input [1:0]state,
    input [1:0]floor,
    output reg beep
    );  
    reg beep_n;//控制蜂鸣声时长
    reg [24:0] beep_cnt=0;
    reg [24:0] clk_500ms_cnt=0;
    reg [24:0] clk_500ms_cnt2=0;   
      always @ (posedge clk)
        begin 
            if(floor==1)
              begin
                clk_500ms_cnt2=0;
                if (clk_500ms_cnt==24999999)//计数0.5s,在此区间发出蜂鸣声
                    begin
                        beep_n=0;
                        clk_500ms_cnt=24999999;
                    end
                else if(clk_500ms_cnt!==24999999&&clk_500ms_cnt!==0)
                    begin
                        clk_500ms_cnt<=clk_500ms_cnt+1;
                        beep_n=1;
                    end  
                else if(clk_500ms_cnt==0)
                    begin
                        clk_500ms_cnt<=clk_500ms_cnt+1;
                        beep_n=0;
                    end
                end 
            else
                begin
                  clk_500ms_cnt=0;
                if (clk_500ms_cnt2==24999999)
                //计数0.5s,在此区间发出蜂鸣声
                    begin
                      beep_n=0;
                      clk_500ms_cnt2=24999999;
                    end
                else if(clk_500ms_cnt2!==24999999&&clk_500ms_cnt2!==0)
                     begin
                        clk_500ms_cnt2<=clk_500ms_cnt2+1;
                        beep_n=1;
                     end  
                else if(clk_500ms_cnt2==0)
                     begin
                        clk_500ms_cnt2<=clk_500ms_cnt2+1;
                        beep_n=0;
                     end
                end 

         if(beep_n==1)
             begin
               if(beep_cnt==49000)
               //向蜂鸣器输入一个约500Hz的脉冲,使蜂鸣器发声
                 begin
                  beep=~beep; 
                  beep_cnt=0;
                 end
                else
                  begin beep_cnt=beep_cnt+1;end 
              end    
         else
             begin
              beep=0;
              beep_cnt=0;          
             end          
    end     
endmodule

//约束文件======================================
set_property IOSTANDARD LVCMOS33 [get_ports {dig[5]}]
set_property IOSTANDARD LVCMOS33 [get_ports {dig[4]}]
set_property IOSTANDARD LVCMOS33 [get_ports {dig[3]}]
set_property IOSTANDARD LVCMOS33 [get_ports {dig[2]}]
set_property IOSTANDARD LVCMOS33 [get_ports {dig[1]}]
set_property IOSTANDARD LVCMOS33 [get_ports {dig[0]}]
set_property IOSTANDARD LVCMOS33 [get_ports {led[3]}]
set_property IOSTANDARD LVCMOS33 [get_ports {led[2]}]
set_property IOSTANDARD LVCMOS33 [get_ports {led[1]}]
set_property IOSTANDARD LVCMOS33 [get_ports {led[0]}]
set_property IOSTANDARD LVCMOS33 [get_ports {seg[7]}]
set_property IOSTANDARD LVCMOS33 [get_ports {seg[6]}]
set_property IOSTANDARD LVCMOS33 [get_ports {seg[5]}]
set_property IOSTANDARD LVCMOS33 [get_ports {seg[4]}]
set_property IOSTANDARD LVCMOS33 [get_ports {seg[3]}]
set_property IOSTANDARD LVCMOS33 [get_ports {seg[2]}]
set_property IOSTANDARD LVCMOS33 [get_ports {seg[1]}]
set_property IOSTANDARD LVCMOS33 [get_ports {seg[0]}]
set_property IOSTANDARD LVCMOS33 [get_ports beep]
set_property IOSTANDARD LVCMOS33 [get_ports clk]
set_property PACKAGE_PIN T5 [get_ports {led[3]}]
set_property PACKAGE_PIN R7 [get_ports {led[2]}]
set_property PACKAGE_PIN R8 [get_ports {led[1]}]
set_property PACKAGE_PIN P9 [get_ports {led[0]}]
set_property PACKAGE_PIN D4 [get_ports clk]
set_property PACKAGE_PIN L2 [get_ports beep]
set_property PACKAGE_PIN N11 [get_ports {dig[5]}]
set_property PACKAGE_PIN N14 [get_ports {dig[4]}]
set_property PACKAGE_PIN N13 [get_ports {dig[3]}]
set_property PACKAGE_PIN M12 [get_ports {dig[2]}]
set_property PACKAGE_PIN H13 [get_ports {dig[1]}]
set_property PACKAGE_PIN G12 [get_ports {dig[0]}]
set_property PACKAGE_PIN L13 [get_ports {seg[7]}]
set_property PACKAGE_PIN M14 [get_ports {seg[6]}]
set_property PACKAGE_PIN P13 [get_ports {seg[5]}]
set_property PACKAGE_PIN K12 [get_ports {seg[4]}]
set_property IOSTANDARD LVCMOS33 [get_ports reset]
set_property IOSTANDARD LVCMOS33 [get_ports start]
set_property PACKAGE_PIN F3 [get_ports reset]
set_property PACKAGE_PIN T9 [get_ports start]
set_property PACKAGE_PIN K13 [get_ports {seg[3]}]
set_property PACKAGE_PIN L14 [get_ports {seg[2]}]
set_property PACKAGE_PIN N12 [get_ports {seg[1]}]
set_property PACKAGE_PIN P11 [get_ports {seg[0]}]
set_property PACKAGE_PIN K3 [get_ports row]
set_property IOSTANDARD LVCMOS33 [get_ports row]
set_property IOSTANDARD LVCMOS33 [get_ports key0]
set_property IOSTANDARD LVCMOS33 [get_ports key1]
set_property IOSTANDARD LVCMOS33 [get_ports key2]
set_property IOSTANDARD LVCMOS33 [get_ports key3]
set_property PACKAGE_PIN R12 [get_ports key0]
set_property PACKAGE_PIN T12 [get_ports key1]
set_property PACKAGE_PIN R11 [get_ports key2]
set_property PACKAGE_PIN T10 [get_ports key3]
set_property IOSTANDARD LVCMOS33 [get_ports {led_b[4]}]
set_property IOSTANDARD LVCMOS33 [get_ports {led_b[3]}]
set_property IOSTANDARD LVCMOS33 [get_ports {led_b[2]}]
set_property IOSTANDARD LVCMOS33 [get_ports {led_b[1]}]
set_property IOSTANDARD LVCMOS33 [get_ports {led_b[0]}]
set_property PACKAGE_PIN T2 [get_ports {led_b[0]}]
set_property PACKAGE_PIN R1 [get_ports {led_b[1]}]
set_property PACKAGE_PIN G5 [get_ports {led_b[2]}]
set_property PACKAGE_PIN H3 [get_ports {led_b[3]}]
set_property PACKAGE_PIN E3 [get_ports {led_b[4]}]

\end{lstlisting}   
    
